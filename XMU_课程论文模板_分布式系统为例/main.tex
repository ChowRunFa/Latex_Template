% !TEX program = xelatex
\documentclass[UTF8]{ctexart}
\usepackage[a4paper,margin=2.5cm]{geometry}
\usepackage{graphicx}
\usepackage{titling}
\usepackage{fancyhdr}
\usepackage{cite}               % 压缩合并数字型引用,如 [1,2,3]→[1–3]
\usepackage{hyperref}
% ———— 用户可修改的信息 ————
\newcommand{\universityimage}{figs/xmu_name.png}
\newcommand{\logoimage}{figs/xmu_seal.png}
\newcommand{\college}{信息学院}
\newcommand{\coursename}{分布式系统}
\newcommand{\reporttype}{课程报告}
\newcommand{\projecttitle}{分布式计算中的负载均衡问题}
\newcommand{\studentname}{张三}
\newcommand{\studentid}{12345678}   % 不需要可删掉
% ————————————————————————

% 页眉页脚设置(正文开始后生效)
\pagestyle{fancy}
\fancyhf{}
\cfoot{\thepage}
\renewcommand{\headrulewidth}{0pt}
\renewcommand{\footrulewidth}{0pt}

% 参考文献标题为“参考文献”
\renewcommand{\refname}{参考文献}

\setlength{\droptitle}{-6em}  % 根据图片大小微调

\begin{document}

%———— 封面 ————
\begin{titlepage}
  \centering
  \includegraphics[width=0.3\textwidth]{\universityimage}\par
  \vspace{1.5cm}
  \includegraphics[width=0.2\textwidth]{\logoimage}\par
  \vspace{1.5cm}
  {\zihao{3}\bfseries \college\par}
  \vspace{1cm}
  {\zihao{3}\bfseries 《\coursename》\reporttype\par}
  \vspace{2cm}
  \begin{tabular}{rl}
    {\zihao{4}题\quad 目:} & {\zihao{4}\projecttitle} \\[1.5em]
    {\zihao{4}姓\quad 名:} & {\zihao{4}\studentname}  \\[1.5em]
    {\zihao{4}学\quad 号:} & {\zihao{4}\studentid}    \\
  \end{tabular}
  \vfill
  {\zihao{4}\the\year\,年\ \the\month\,月\ \the\day\,日\par}
  \thispagestyle{empty}
\end{titlepage}

%———— 摘要 ————
\clearpage
\section*{摘要}
这里填写摘要内容,字数一般200–300字。摘要应简要介绍研究背景、方法、结果和结论。
\thispagestyle{empty}

%———— 目录 ————
\clearpage
\phantomsection
\tableofcontents
\thispagestyle{empty}

%———— 正文 ————
\clearpage
\pagenumbering{arabic}
\section{引言}
本报告主要讨论分布式计算中的负载均衡问题。已有研究表明\cite{Smith2021},合理的负载策略可以显著提升系统吞吐量……

% … 其他章节 …

%———— 参考文献 ————
\clearpage
\bibliographystyle{unsrt}       % 按引用顺序编号
\bibliography{references}       % references.bib 文件(不带 .bib 后缀)

\end{document}
